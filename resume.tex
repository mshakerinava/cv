\documentclass{resume}

%%%%%%%%%%%%%%%%
%%% Packages %%%
%%%%%%%%%%%%%%%%

\usepackage{url}
\usepackage[utf8]{inputenc}
\usepackage{textcomp}
\usepackage{hyperref}
\usepackage{multicol}
\usepackage{lmodern}
\usepackage{bold-extra}
\usepackage{amssymb}

\usepackage[top=0in, left=1.5in, bottom=1.5in, right=1.5in]{geometry}

%%%%%%%%%%%%%%%%
%%% Commands %%%
%%%%%%%%%%%%%%%%

\renewcommand{\familydefault}{\rmdefault}
\renewcommand{\categoryfont}{\sc}
% \renewcommand{\titlenamefont}{\LARGE\sc\textbf{}}

\renewcommand{\labelcitem}{$\diamond$}
\renewcommand{\labelitemi}{\textbullet}
\newcommand{\first}{$\mbox{1}^{\mbox{\scriptsize st}}$\ }
\newcommand{\second}{$\mbox{2}^{\mbox{\scriptsize nd}}$\ }
\newcommand{\third}{$\mbox{3}^{\mbox{\scriptsize rd}}$\ }
\newcommand{\fourth}{$\mbox{4}^{\mbox{\scriptsize th}}$\ }
\newcommand{\fifth}{$\mbox{5}^{\mbox{\scriptsize th}}$\ }
\newcommand{\seventh}{$\mbox{7}^{\mbox{\scriptsize th}}$\ }
\newcommand{\InternalSpace}{\vspace{0.18cm}}
\newcommand{\ExternalSpace}{\vspace{-0.09cm}}

\renewcommand{\textapprox}{\raisebox{0.5ex}{\texttildelow}}

% My Environments and Commands:

\newenvironment{MySection}[1]
{\begin{category}{#1}}
{\end{category}}

\newcommand{\MyItem}{\citembullet}

% Template

\setlength{\oddsidemargin}{1.75cm}
\setlength{\marginparwidth}{3cm}\addtolength{\marginparwidth}{-\marginparsep}
\setlength{\evensidemargin}{\oddsidemargin}
\setlength{\textwidth}{\paperwidth}
\addtolength{\textwidth}{-7.2cm}
\addtolength{\textwidth}{-1.25\oddsidemargin}
\addtolength{\textwidth}{\marginparwidth}
\addtolength{\textwidth}{\marginparsep}
\setlength{\topmargin}{-2.0cm}
\addtolength{\textheight}{-1.5cm} % specifies bottom margin?


% Months:
\def \January   { Jan.  }
\def \February  { Feb.  }
\def \March     { Mar.  }
\def \April     { Apr.  }
\def \May       { May   }
\def \June      { June  }
\def \July      { July  }
\def \August    { Aug.  }
\def \September { Sept. }
\def \October   { Oct.  }
\def \November  { Nov.  }
\def \December  { Dec.  }


%%%%%%%%%%%%%%%
%%% Defines %%%
%%%%%%%%%%%%%%%

\def \StudentName {Mehran Shakerinava}

\def \CurrentDepartmentName    {School of Computer Science}
\def \CurrentUniversityName    {McGill University}
\def \CurrentUniversityAddress {3480 University St., Montréal, Québec}
\def \CurrentPostalCode        {H3A 0E9}

\def \FirstEmail  {mehran.shakerinava@mail.mcgill.ca}
\def \SecondEmail {mehran.shakerinava@mila.quebec}

\def \Homepage {https://mila.quebec/en/person/mehran-shakerinava/}

\def \CellPhone {+1(438)881-3645}

\def \YearOfBirth {1995}
\def \MonthOfBirth {\November}
\def \DayOfBirth {4}

\def \ISCAXIX {\href{https://iscaconf.org/isca2019/}{ISCA 2019}}
\def \HPCAXIX {\href{http://hpca2019.seas.gwu.edu/}{HPCA 2019}}
\def \DPCIII  {\href{https://dpc3.compas.cs.stonybrook.edu/}{The Third Data Prefetching Championship (DPC3)}}

\def \Lotfi {\href{https://scholar.google.com/citations?user=UZIKgasAAAAJ}{Pejman Lotfi-Kamran}}
\def \Sarbazi {\href{https://scholar.google.com/citations?user=9OHC9AsAAAAJ}{Hamid Sarbazi-Azad}}
\def \Soleymani {\href{https://scholar.google.com/citations?user=S1U0KlgAAAAJ}{Mahdieh Soleymani Baghshah}}
\def \Siamak {\href{https://www.siamak.page}{Siamak Ravanbakhsh}}

\InternalSpace
\InternalSpace

%%%%%%%%%%

\centering\author{\StudentName}

%%%%%%%%%%

\address{
\CurrentDepartmentName\\
\CurrentUniversityName\\
\CurrentUniversityAddress\\
\CurrentPostalCode\\
}
{
%Emails and Homepage:
\texttt{\FirstEmail}\\
\texttt{\SecondEmail}\\
% \url{\Homepage}\\
%Personal Information
\texttt{\CellPhone}
% Date of Birth: \MonthOfBirth\ \DayOfBirth, \YearOfBirth\\
}


\def\shr{\CurrentUniversityName}

%%%%%%%%%%

% \usepackage{geometry}
% \geometry{a4paper, total={210mm, 297mm}, left=50mm, right=25mm, top=25mm, bottom=40mm}

\begin{document}

\maketitle

\vspace{-0.75cm}

\InternalSpace
\InternalSpace


%%%%%%%%%%%%%%%%%
%%% Education %%%
%%%%%%%%%%%%%%%%%

\begin{MySection}{Education}

\MyItem
\textbf{Ph.D. in Computer Science} \hfill{ 2021 -- Now }\\
\href{https://mila.quebec/en/}{Mila} and McGill University, Montréal, Québec\\
Advised by \Siamak\\
\textsc{GPA}: $4/4$

\MyItem
\textbf{M.Sc. in Computer Science} \hfill{ 2020 -- 2021 }\\
\href{https://mila.quebec/en/}{Mila} and McGill University, Montréal, Québec\\
Advised by \Siamak\\
Fast-tracked to Ph.D.

\MyItem
\textbf{B.Sc. in Computer Engineering} \hfill{ 2014 -- 2019 }\\
Sharif University of Technology, Tehran, Iran\\
\textsc{GPA}: $17.41/20$ (150 credits)\\

\end{MySection}

\InternalSpace


%%%%%%%%%%%%%%%%%%%%%%%%%%
%%% Research Interests %%%
%%%%%%%%%%%%%%%%%%%%%%%%%%

\begin{MySection}{Research Interests}

% \MyItem \textbf{Computer Architecture}: \emph{Speculation, High-Performance Computing, Parallelism}

% \MyItem Robot learning
% \MyItem Deep reinforcement learning
% \MyItem Differentiable self-organizing systems
\MyItem Combinatorial Problems and Algorithms in Machine Learning, Optimization, and Statistics
\MyItem Symmetry and Equivariance
\MyItem Game Theory and Reinforcement Learning

\end{MySection}

\InternalSpace


%%%%%%%%%%%%%%%%%%%%%%%%%%%
%%% Research Experience %%%
%%%%%%%%%%%%%%%%%%%%%%%%%%%

% \begin{MySection}{Research Experience}

% \MyItem \textbf{\href{http://rl.cs.mcgill.ca/}{Reasoning and Learning Lab (RLLAB)}}\\
% Graduate Research Assistant\\
% Advisor: \Siamak

% \begin{itemize}

% \item \textbf{Learning on Spherical Data}:
% I'm currently working on learning from data that lies on the surface of a sphere.
% Examples of such data include Cosmological Microwave Background (CMB) maps, omni-directional images, and global climate records.
% We model the sphere as a platonic solid with pixelized faces.
% By combining the symmetry of the platonic solid with independendent translation symmetry of the face pixelization, we obtain a surrogate symmetry for the whole sphere.
% We then design a linear layer that respects this symmetry and show that a deep neural network with this kind of layer is able to achive
% state-of-the-art results on omni-directional image segmentation and segmentation of extreme climate events.

% \item \textbf{Equivariant Compression}:
% We are investigating layers that are able to reduce the dimension of data while exactly preserving
% the inherent symmetries. Such layers may provide a mathematically principled approach to
% disentangling factors of variation.


% \end{itemize}

% \InternalSpace\ExternalSpace

% \MyItem \textbf{\href{http://mll.ce.sharif.edu}{Machine Learning Laboratory (MLL)}}\\
% Undergraduate Research Assistant\\
% Advisor: \Soleymani

% \begin{itemize}

% \item \textbf{Deep Reinforcement Learning}:
% I Assisted a senior M.Sc. student in his thesis research.
% I implemented several deep reinforcement learning agents and trained them to play Atari.
% I also trained environment models using data gathered from trained agents.
% We also performed many experiments on intrinsic motivation and exploration in deep reinforcement learning.

% \item \textbf{Graph Representation Learning}:
% We interpreted the commute distance of nodes in a graph as a measure of locality
% and designed a convolutional layer that is local with respect to this measure.
% We also theoretically proved that such a graph convolution layer is $\epsilon$-spectral.
% To process signals on a given graph, first,
% the commute distance is obtained in a preprocessing step via spectral graph embedding.
% Then, we construct a convolutional neural network based on the obtained node embedding.
% I implemented and evaluated this method on various datasets.

% \end{itemize}

% \InternalSpace\ExternalSpace

% \MyItem \textbf{\href{http://hpcan.ce.sharif.edu}{High Performance Computing Architectures and Networks Laboratory (HPCAN)}}\\
% %Sharif University of Technology, Tehran, Iran (Fall 2017 -- Present).\\
% Undergraduate Research Assistant\\
% Advisors: \Sarbazi\ and \Lotfi

% \begin{itemize}

% \item \textbf{Data Prefetching}: Memory access is the performance bottleneck in most modern computer systems, and therefore, reducing the number of cache misses is of paramount importance.
% One widely used approach in this regard is data prefetching.
% Data prefetching consists of predicting future memory requests and (pre-)fetching those memory blocks into the cache, before they are explicitly requested by the processor.
% In this work, I implemented and helped develop several novel algorithms for data prefetching.
% One of these prefetchers, named Bingo, is the current state-of-the-art for prefetching in multi-core systems.
% Bingo was published in \textit{HPCA}, which is one of the most prestigious conferences in computer architecture.

% \end{itemize}

% \end{MySection}

% \InternalSpace


%%%%%%%%%%%%%%%%%%%%
%%% Publications %%%
%%%%%%%%%%%%%%%%%%%%

\begin{MySection}{Publications}

\MyItem \textbf{M.~Shakerinava}, A. K.~Mondal, and S.~Ravanbakhsh. {``Structuring Representations Using Geometric Invariants."} \textit{Advances in Neural Information Processing Systems (NeurIPS)}, \href{https://papers.nips.cc/paper_files/paper/2022/hash/dcd297696d0bb304ba426b3c5a679c37-Abstract-Conference.html}{2022}.

\MyItem \textbf{M.~Shakerinava}, and S.~Ravanbakhsh. {``Utility Theory for Sequential Decision Making.''} \textit{International Conference on Machine Learning (ICML)}. PMLR, \href{https://proceedings.mlr.press/v162/shakerinava22a/shakerinava22a.pdf}{2022}.

\MyItem \textbf{M.~Shakerinava}, and S.~Ravanbakhsh. {``Equivariant networks for pixelized spheres."} \textit{International Conference on Machine Learning (ICML)}. PMLR, \href{http://proceedings.mlr.press/v139/shakerinava21a/shakerinava21a.pdf}{2021}.

\MyItem F.~Golshan, M.~Bakhshalipour, \textbf{M.~Shakerinava}, A.~Ansari, P.~Lotfi-Kamran, and H.~Sarbazi-Azad, {``Harnessing Pairwise-Correlating Data Prefetching With Runahead Metadata.''} \textit{Computer Architecture Letters (CAL)}, \href{https://ieeexplore.ieee.org/abstract/document/9177277}{2020}.

\MyItem M.~Bakhshalipour, \textbf{M.~Shakerinava}, P.~Lotfi-Kamran, and H.~Sarbazi-Azad, {``Bingo Spatial Data Prefetcher.''} \textit{International Symposium on High-Performance Computer Architecture (HPCA)}, \href{https://cs.ipm.ac.ir/~plotfi/papers/bingo_hpca19.pdf}{2019}.

\end{MySection}

\InternalSpace

\begin{MySection}{Workshop Papers}

\MyItem \textbf{M.~Shakerinava}, M.~Bakhshalipour, P.~Lotfi-Kamran, and H.~Sarbazi-Azad, {``Multi-Lookahead Offset Prefetching,''} in \textit{The Third Data Prefetching Championship (DPC3)}, in conjunction with \textit{International Symposium on Computer Architecture (ISCA)}, \href{https://dpc3.compas.cs.stonybrook.edu/pdfs/Multi_lookahead.pdf}{2019}.
\MyItem M.~Bakhshalipour, \textbf{M.~Shakerinava}, P.~Lotfi-Kamran, and H.~Sarbazi-Azad, {``Accurately and Maximally Prefetching Spatial Data Access Patterns with Bingo,''} in \textit{The Third Data Prefetching Championship (DPC3)}, in conjunction with \textit{International Symposium on Computer Architecture (ISCA)}, \href{https://dpc3.compas.cs.stonybrook.edu/pdfs/Accurately.pdf}{2019}.

\end{MySection}

\begin{MySection}{Preprints}

\MyItem M.~Bakhshalipour, \textbf{M.~Shakerinava}, F.~Golshan, A.~Ansari, P.~Lotfi-Karman, and H.~Sarbazi-Azad, {``A Survey on Recent Hardware Data Prefetching Approaches with An Emphasis on Servers.''} \textit{arXiv preprint arXiv:2009.00715}, \href{https://arxiv.org/pdf/2009.00715.pdf}{2020}.

\end{MySection}

\InternalSpace

%%%%%%%%%%%%%%%%%%%%%%%
%%% Honors & Awards %%%
%%%%%%%%%%%%%%%%%%%%%%%

\begin{MySection}{Honors and Awards}

\MyItem
Kharusi Family International Science Fellowship \hfill {2020}\\
\emph{McGill University}\\
Valued at 7500\$

\MyItem
\textbf{1st Place} (among \textapprox30,000 participants) \hfill {2019}\\
\emph{Iran's National Master's Entrance Exam}\\
Computer Engineering track, AI/Robotics major

\MyItem
\textbf{2nd and 3rd Place (1st and 2nd Place in Multi-Core Setting)} \hfill {2019}\\
\emph{\DPCIII\ at \ISCAXIX}\\

\MyItem
\textbf{1st Place} \hfill {2016}\\
Programming Contest at Iran's 3rd Python Conference (\emph{PyCon 2016})

\MyItem
\textbf{Silver Medal} (Ranked \textapprox15 among \textapprox10,000) \hfill {2012}\\
21st \emph{Iranian National Olympiad in Informatics}

\end{MySection}

\InternalSpace


%%%%%%%%%%%%%%%%%%%%%%%%%%%
%%% Teaching Experience %%%
%%%%%%%%%%%%%%%%%%%%%%%%%%%

\begin{MySection}{Teaching Experience}

\MyItem \textbf{Organizing Team}\\
\InternalSpace\ExternalSpace
\textit{Mathematics Study Group} (Mila Quebec AI Institute) \hfill 2023\\
Taught real mathematical analysis and held problem-solving sessions.
\InternalSpace\ExternalSpace

\MyItem \textbf{Teaching Assistant}\\
\InternalSpace\ExternalSpace
\textit{Algorithms and Data Structures} (McGill University) \hfill Fall 2023\\
\InternalSpace\ExternalSpace
\textit{Theory of Computation} (McGill University) \hfill Fall 2022\\
\InternalSpace\ExternalSpace
\textit{Probabilistic Graphical Models} (McGill University) \hfill Winter 2022\\
\InternalSpace\ExternalSpace
\textit{Artificial Intelligence} (Sharif University of Technology) \hfill Spring 2018 and Spring 2019\\
\InternalSpace\ExternalSpace
\textit{Advanced Programming} (Sharif University of Technology) \hfill Fall 2016\\
\InternalSpace\ExternalSpace

\MyItem \textbf{Teacher}\\ 
\InternalSpace\ExternalSpace
\textit{Informatics Olympiad} (\href{https://en.wikipedia.org/wiki/National_Organization_for_Development_of_Exceptional_Talents}{NODET} High-School) \hfill{2012 -- 2015}\\
Taught topics on Combinatorics, Graph Theory, Algorithms, and Programming.
\InternalSpace\ExternalSpace

\end{MySection}

\InternalSpace


%%%%%%%%%%%%%%%%%%%%%%%
%%% Work Experience %%%
%%%%%%%%%%%%%%%%%%%%%%%

% Do not include! Might cause VISA problems!

% \begin{MySection}{Work Experience}

% \MyItem \textbf{\href{http://lyan.co}{Lyan} (currently \href{https://shopgramapp.com}{Shopgram})}\\
% Azadi Avenue, Tehran, Iran (2015 -- 2016).\\
% Back-end web developer\\

% \end{MySection}

% \InternalSpace


%%%%%%%%%%%%%%%%%%%%%%%%
%%% Selected Courses %%%
%%%%%%%%%%%%%%%%%%%%%%%%

% \begin{MySection}{Selected Courses}

% \MyItem \textbf{B.Sc., Sharif University of Technology}:

% \begin{multicols}{2}

% \begin{itemize}
% \item {Deep Learning (18.6/20)}

% \item {Linear Algebra (19.6/20)}

% % \item {Electrical Circuits (19.2/20)}

% \item {Automata and Compilers (20/20)}
% % \item {Digital Systems Design (20/20)}

% % \item {Computer Architecture (19.7/20)}
% \item {Data Structures and Algorithms (20/20)}

% \item {Advanced Programming (19.5/20)}
% \item {Multivariate Calculus (20/20)}

% % \item {Intro to Programming (20/20)}
% \end{itemize}

% \columnbreak

% \begin{itemize}
% \item {B.Sc. Thesis (20/20)}

% \item {Foundations of Neuroscience (17.8/20)}
% % \item {Advanced Logic Design (20/20)}
% % \item {Computer Interface Circuits (19.9/20)}

% % \item {VLSI Design (18.4/20)}
% \item {Real-Time Systems (18.2/20)}
% \item {Signals and Systems (19/20)}

% \item {Artificial Intelligence (18.6/20)}
% \item {Computer Networks (18.7/20)}
% \end{itemize}

% \end{multicols}

% \end{MySection}

% \InternalSpace


%%%%%%%%%%%%%%%%%%%%%%%%%
%%% Selected Projects %%%
%%%%%%%%%%%%%%%%%%%%%%%%%

% \begin{MySection}{Selected Projects}

% \MyItem \textbf{Deep Reinforcement Learning Assignments}\footnote{\href{https://github.com/mshakerinava/berkeleydeeprlcourse-solutions/}{https://github.com/mshakerinava/berkeleydeeprlcourse-solutions/}}\\
% \InternalSpace\ExternalSpace
% Solved the assignments for \href{http://rail.eecs.berkeley.edu/deeprlcourse/}{Berkeley's Deep RL course}. The solutions cover Imitation Learning, Policy Gradients, Q-Learning, Actor-Critic, Model-Based RL, and Exploration.
% \InternalSpace\ExternalSpace

% \MyItem \textbf{Variational Autoencoder Library}\footnote{\href{https://github.com/mshakerinava/vae-lib/}{https://github.com/mshakerinava/vae-lib/}}\\
% \InternalSpace\ExternalSpace
% Wrote a modular VAE library for visualization and drop-in use in Deep Learning projects.
% \InternalSpace\ExternalSpace

% \MyItem \textbf{CUDA K-Means}\\
% \textit{Winter 2019 - Multi-Core Computing}\\
% \InternalSpace\ExternalSpace
% Implemented a highly efficient CUDA kernel for the k-means clustering algorithm.
% \InternalSpace\ExternalSpace

% \MyItem \textbf{Video Streaming WiFi Robot}\\
% \textit{Winter 2018 - Hardware Lab}\\
% \InternalSpace\ExternalSpace
% Assembled and programmed a WiFi-controlled \href{https://www.arduino.cc/}{Arduino} robot with a camera. The video is streamed to a responsive web-based UI from which the robot can be remotely controlled.
% \InternalSpace\ExternalSpace

% % \MyItem \textbf{Network Protocols}\\
% % \textit{Winter 2017 - Computer Networks Programming Assignments}\\
% % \InternalSpace\ExternalSpace
% % Implemented DHCP, TCP, and BGP in Java.
% % \InternalSpace\ExternalSpace

% % \MyItem \textbf{Local Search}\\
% % \textit{Winter 2017 - Artificial Intelligence Programming Assignment}\\
% % \InternalSpace\ExternalSpace
% % Implemented Hill-climbing, Simulated Annealing, and Genetic Algorithm to solve a discrete optimization problem.
% % \InternalSpace\ExternalSpace

% \MyItem \textbf{Cython Compiler}\\
% \textit{Fall 2016 - Automata and Compilers Project}\\
% \InternalSpace\ExternalSpace
% Wrote a compiler for a made-up programming language called Cython. The compiler used LALR(1) parsing and was written in C\texttt{++} with \href{https://github.com/westes/flex}{Flex}, \href{https://www.gnu.org/software/bison}{Bison}, and \href{https://llvm.org}{LLVM}.
% \InternalSpace\ExternalSpace

% \MyItem \textbf{FPGA Odometry}\\
% \textit{Fall 2016 - Digital Systems Design Project}\\
% \InternalSpace\ExternalSpace
% Implemented \href{https://en.wikipedia.org/wiki/Odometry}{odometry} in C on Arduino and afterward, in Verilog on FPGA. Constructed a \href{https://en.wikipedia.org/wiki/Differential_wheeled_robot}{differential drive robot} with optical wheel encoders (consisting of encoder disks and photoelectric sensors), and tested the implementation on it successfully. \textit{(Best Class Project)}
% \InternalSpace\ExternalSpace

% \MyItem \textbf{Pipelined Processor}\\
% \textit{Winter 2016 - Computer Architecture Bonus Project}\\
% \InternalSpace\ExternalSpace
% Verilog implementation of the 5-stage \href{https://en.wikipedia.org/wiki/Classic_RISC_pipeline}{classic RISC pipeline}.
% \InternalSpace\ExternalSpace

% % \MyItem \textbf{Pacman Game}\\
% % \textit{Fall 2015 - Advanced Programming Final Project}\\
% % \InternalSpace\ExternalSpace
% % A clone of classic \href{https://en.wikipedia.org/wiki/Pac-Man}{Pacman} written in C\texttt{++} with \href{http://www.cocos2d-x.org/}{Cocos2d-x}.
% % \InternalSpace\ExternalSpace

% % \MyItem \textbf{15-Puzzle AI and GUI}\\
% % \textit{Fall 2015 - Advanced Programming Midterm Project}\\
% % \InternalSpace\ExternalSpace
% % A GUI for a \href{https://en.wikipedia.org/wiki/15_puzzle}{15-Puzzle} game, and an AI that can solve it. Written in C\texttt{++} with \href{https://qt.io}{Qt}.
% % \InternalSpace\ExternalSpace

% \MyItem \textbf{Bare Metal Raspberry Pi Programming}\\
% \textit{Fall 2015 - Computer Structure and Language Project}\\
% \InternalSpace\ExternalSpace
% Programmed a \href{https://raspberrypi.org}{Raspberry Pi} in ARM assembly to draw the Sierpinski triangle with a \href{https://en.wikipedia.org/wiki/Rule_90}{Rule 90} cellular automaton. \textit{(Best class project)}
% \InternalSpace\ExternalSpace

% % \MyItem \textbf{Flight Control Game}\\
% % \textit{Fall 2014 - Intro to Programming Project}\\
% % \InternalSpace\ExternalSpace
% % A clone of \href{https://en.wikipedia.org/wiki/Flight_Control_(video_game)}{Flight Control} written in C with \href{https://gtk.org}{GTK}.
% % \InternalSpace\ExternalSpace

% \MyItem \textbf{2048 AI}\\
% \textit{Fall 2014 - Intro to Programming Open-ended Homework Problem}\\
% \InternalSpace\ExternalSpace
% An AI for the game \href{http://gabrielecirulli.github.io/2048}{2048}, based on Monte-Carlo tree search and heuristics. It's able to reach 4096.
% \InternalSpace\ExternalSpace

% \end{MySection}

% \InternalSpace


%%%%%%%%%%%%%%%%%%%%%%
%%% Voluntary Work %%%
%%%%%%%%%%%%%%%%%%%%%%

% \begin{MySection}{Voluntary Work}

% \MyItem \textbf{8th and 9th \href{https://aichallenge.sharif.edu}{Sharif AI Challenge}}\\
% Sharif University of Technology, Tehran, Iran (2016 and 2017)\\
% Technical Staff (C\texttt{++} Client)\\

% \end{MySection}

% \InternalSpace


%%%%%%%%%%%%%%
%%% Skills %%%
%%%%%%%%%%%%%%

\begin{MySection}{Skills}

\MyItem 
C/C\texttt{++}, Java, Python, MATLAB, PyTorch, TensorFlow, CUDA, Git, Verilog, JavaScript, \LaTeX

\end{MySection}

\InternalSpace


%%%%%%%%%%%%%%%%%%%%%%%%%%%%
%%% Professional Service %%%
%%%%%%%%%%%%%%%%%%%%%%%%%%%%

\begin{MySection}{Professional Service}

\MyItem \textbf{Reviewer} \\ NeurIPS 2022-2023, ICLR 2024, AISTATS 2024, Montreal AI Symposium (MAIS) 2022.

\end{MySection}

\InternalSpace


%%%%%%%%%%%%%%
%%% Skills %%%
%%%%%%%%%%%%%%

\begin{MySection}{References}

\MyItem \textbf{Siamak Ravanbakhsh (Advisor)}\\
Assistant Professor, School of Computer Science, McGill University\\
\texttt{siamak@cs.mcgill.ca}

\end{MySection}

\InternalSpace



%%%%%%%%%%%%%%%%%
%%% Languages %%%
%%%%%%%%%%%%%%%%%

% \begin{MySection}{Languages}

% \MyItem Persian (Mother Tongue)
% \MyItem English (Fluent)

% \begin{itemize}

% \item {TOEFL (Reading: 30, Listening: 30, Speaking: 27, Writing: 26)}
% \item {GRE (Quantitative: 170, Verbal: 162, Writing: 3.5)}

% \end{itemize}

% \MyItem Swedish (Basic)

% \end{MySection}


\end{document}
